\documentclass[11pt,oneside]{article}

% This package simply sets the margins to be 1 inch.
\usepackage[margin=1in]{geometry}

% This package makes code look pretty
\usepackage{listings}

% Include Graphics
\usepackage{graphicx}
\usepackage{subfig}

% Put pictures and tables in their correct place.
\usepackage{float}

% used for wrapping text around figures
\usepackage{wrapfig}

% Package for creating table
\usepackage{multirow}

% This removes page numbers.
%\pagenumbering{gobble}

% These packages include nice commands from AMS-LaTeX
\usepackage{amssymb,amsmath,amsthm}

% Make the space between lines slightly more
% generous than normal single spacing, but compensate
% so that the spacing between rows of matrices still
% looks normal.  Note that 1.1=1/.9090909...
\renewcommand{\baselinestretch}{1.1}
\renewcommand{\arraystretch}{.91}

% Define an environment for exercises.
\newenvironment{exercise}[1]{\vspace{.5cm}\noindent\textbf{#1 \hspace{.05em}}}{}

% Allow for underlining.
\usepackage[normalem]{ulem}
\usepackage{multirow}

% define shortcut commands for commonly used symbols
\newcommand{\R}{\mathbb{R}}
\newcommand{\C}{\mathbb{C}}
\newcommand{\Z}{\mathbb{Z}}
\newcommand{\Q}{\mathbb{Q}}
\newcommand{\N}{\mathbb{N}}
\newcommand{\calP}{\mathcal{P}}


\DeclareMathOperator{\vsspan}{span}



\usepackage{Sweave}
\begin{document}
\Sconcordance{concordance:RMySQL_setup.tex:RMySQL_setup.Rnw:%
1 54 1 1 0 68 1 1 2 1 0 1 4 6 0 1 2 4 1 1 2 6 0 1 1 6 0 1 2 4 1 1 2 1 0 %
1 1 12 0 1 2 4 1 1 2 1 0 3 1 9 0 1 5 9 0 1 2 4 1 1 5 4 0 1 1 5 0 1 2 1 %
1 9 0 1 2 3 0 1 2 2 1}


\begin{flushright}
David Ebert\\
\today \\
\end{flushright}

\begin{center}
\textbf{Tutorial: Connecting R and MySQL} \\
\end{center}

Here are the steps I used to combine R and MySQL together happily.

\begin{enumerate}

\item Get MySQL and the R package RMySQL.



\item Create a database in MySQL. Notice that while the syntax for \texttt{CREATE TABLE} is the same as Oracle, the data types are different. In particular, MySQL uses \texttt{TINYINT}, \texttt{SMALLINT}, etc. for integers.

\begin{verbatim}
$ mysql -u root -p -h localhost

mysql> CREATE DATABASE shop;

mysql> CREATE TABLE inventory
        (id SMALLINT PRIMARY KEY, 
        name VARCHAR(50),
        quantity SMALLINT);

mysql> INSERT INTO inventory VALUES(1, 'Tomato', '10');
mysql> INSERT INTO inventory VALUES(2, 'Potato', '20');
mysql> INSERT INTO inventory VALUES(3, 'Rhubarb', '0');
mysql> INSERT INTO inventory VALUES(4, 'Eggplant', '2');
mysql> INSERT INTO inventory VALUES(5, 'Brussels Spouts', '15');
mysql> INSERT INTO inventory VALUES(6, 'Onion', '10');

mysql> SELECT * from inventory;

mysql> DESCRIBE inventory; -- This is a lot like the schema tab in Oracle SQL live
\end{verbatim}




\item Next, we need to create a user with permission to access the database. To do that, I created a new user in MySQL.\footnote{I'm sure it's possible to skip this step and use root user. But I had trouble with this in R. I guess this workaround is good DBA practice.} 

\begin{verbatim}
$ mysql -u root -p -h localhost

mysql> CREATE USER 'dangle'@'localhost' IDENTIFIED BY 'dongle'; 
mysql>   -- username is dangle. password is dongle.
mysql> GRANT ALL PRIVILEGES ON shop.* TO 'francesco'@'localhost';
mysql> quit;
\end{verbatim}


We can now access the database directly through R. But if you want to log in to MySQL as dangle, you would use the following command before entering the password:

\begin{verbatim}
$ mysql -u dangle -p -h localhost
\end{verbatim}



\item We're now ready to use R. First, we need to create a database connection, which we'll name \texttt{dbcon}:

\begin{Schunk}
\begin{Sinput}
> library(RMySQL)
> dbcon = dbConnect(MySQL(), user = 'dangle', 
+                  password = 'dongle', 
+                  dbname = 'shop', 
+                  host = '127.0.0.1')
\end{Sinput}
\end{Schunk}



\item Now we're ready. First, let's look at the tables in the database, and then the columns in the inventory.

\begin{Schunk}
\begin{Sinput}
> dbListTables(dbcon)
\end{Sinput}
\begin{Soutput}
[1] "employee"  "inventory"
\end{Soutput}
\begin{Sinput}
> dbListFields(dbcon, 'inventory')
\end{Sinput}
\begin{Soutput}
[1] "id"       "name"     "quantity"
\end{Soutput}
\end{Schunk}


\vspace{25 pt}
Next, let's bring the table data into R. 

\begin{Schunk}
\begin{Sinput}
> inventory_table = dbReadTable(dbcon, "inventory")
> inventory_table
\end{Sinput}
\begin{Soutput}
  id            name quantity
1  1          Tomato       10
2  2          Potato       20
3  3         Rhubarb        0
4  4        Eggplant        2
5  5 Brussels Spouts       15
6  6           Onion       10
\end{Soutput}
\end{Schunk}


\vspace{25 pt}
We can go the other way, too. Let's create a new table in R, called \texttt{employee} and send that table to the \texttt{shop} database.

\begin{Schunk}
\begin{Sinput}
> fname = c("Alice", "Bob", "Charlie", "Dave")
> lname = c("Alvarez", "Brown", "Chaplin", "Dangle")
> employee = data.frame(lname, fname)
> employee
\end{Sinput}
\begin{Soutput}
    lname   fname
1 Alvarez   Alice
2   Brown     Bob
3 Chaplin Charlie
4  Dangle    Dave
\end{Soutput}
\begin{Sinput}
> dbWriteTable(dbcon, "employee", 
+              employee, 
+              overwrite = TRUE, 
+              append = FALSE)
\end{Sinput}
\begin{Soutput}
[1] TRUE
\end{Soutput}
\end{Schunk}


\vspace{25 pt}
Finally, let's send queries to the database from within R. First we send a query to SQL. Then the \texttt{fetch} command imports the table into R. 

\begin{Schunk}
\begin{Sinput}
> sql_query = dbSendQuery(dbcon, 
+                         'SELECT name, quantity 
+                         FROM inventory 
+                         WHERE quantity > 8')
> sql_query
\end{Sinput}
\begin{Soutput}
<MySQLResult:8,0,8>
\end{Soutput}
\begin{Sinput}
> inventory_table = fetch(sql_query, n=-1)
> inventory_table
\end{Sinput}
\begin{Soutput}
             name quantity
1          Tomato       10
2          Potato       20
3 Brussels Spouts       15
4           Onion       10
\end{Soutput}
\begin{Sinput}
> on.exit(dbDisconnect(dbcon))
\end{Sinput}
\end{Schunk}
\end{enumerate}

\end{document}
